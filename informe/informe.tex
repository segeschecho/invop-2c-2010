 %%	SECCION documentclass																									 %%	
%%---------------------------------------------------------------------------%%
\documentclass[a4paper]{report}

%%---------------------------------------------------------------------------%%
%%	SECCION usepackage																											 %%	
%%---------------------------------------------------------------------------%%
\usepackage{amsmath, amsthm}
\usepackage[spanish,activeacute]{babel}
\usepackage{caratula}
\usepackage{a4wide}
\usepackage{hyperref}
\usepackage{fancyhdr}
\usepackage{graphicx} % Para el logo magico!
\usepackage{amssymb}
\usepackage{amsmath}
\usepackage{float}
\usepackage[latin1]{inputenc}
%\usepackage [T1]{fontenc}
\usepackage[dvipsnames,usenames]{color}
\usepackage{amsfonts}
\usepackage{ulem}
%\usepackage{highlight}
\usepackage{fancybox}
%\usepackage{marvosym}
\usepackage{color}
\usepackage{lastpage}
\usepackage{lscape}
\usepackage{tabularx}
\usepackage{algorithmic}
\usepackage{algorithm}
\usepackage{subfigure}

%%---------------------------------------------------------------------------%%
%%	SECCION opciones																												 %%	
%%---------------------------------------------------------------------------%%
\parskip    = 11 pt
\headheight	= 13.1pt
\pagestyle	{fancy}
\definecolor{orange}{rgb}{1,0.5,0}

\addtolength{\headwidth}{1.0in}

\addtolength{\oddsidemargin}{-0.5in}
\addtolength{\textwidth}{1.0in}
\addtolength{\topmargin}{-0.5in}
\addtolength{\textheight}{0.7in}

%%---------------------------------------------------------------------------%%
%%	SECCION document	 %%	
%%---------------------------------------------------------------------------%%
\begin{document}
\renewcommand{\chaptername}{Parte }
\input{trans_algorithmic.tex}
%%---- Caratula -------------------------------------------------------------%%
\materia{Investigaci�n Operativa (2do cuatrimestre de 2010)}
\titulo{Trabajo Pr�ctico}

\integrante{Gonzalez, Emiliano}{426/06}{xjesse\_jamesx@hotmail.com}
\integrante{Gonzalez, Sergio}{481/06}{gonzalezsergio2003@yahoo.com.ar}
\resumen{
En el siguiente documento, se mostrar�n diferentes pruebas de performance utilizando la herramienta CPLEX. Las pruebas consisten en la modificaci�n de diferentes t�cnicas de brancheo del m�todo Branch and Bound y del m�todo Branch and Cut, esta �ltima con una parte implementada especialmente para resolver el problema de separaci�n.}

% TOC, usa estilos locos
\maketitle
\pagestyle{empty}
{
\fancypagestyle{plain}
    {
    \fancyhead{}
    \fancyfoot{}
    \renewcommand{\headrulewidth}{0.0pt}
    } % clear header and footer of plain page because of ToC
\tableofcontents
}

\newpage
% arreglos los estilos para el resto del documento, y
% reseteo los numeros de pagina para que queden bien
\pagenumbering{arabic}
\fancypagestyle{plain} {
    \fancyhead[LO]{Gonz�lez, Gonz�lez}
    \fancyhead[C]{}
    \fancyhead[RO]{P\'agina \thepage\ de \pageref{LastPage}}
    \fancyfoot{}
    \renewcommand{\headrulewidth}{0.4pt}
}
\pagestyle{plain}

\newpage
\section{Ejercicio1}

El objetivo de este ejercicio consiste en poner a prueba el problema de coloreo, variando algunos de los par�metros que $Cplex$ utiliza para resolver el problema. Los mismos son par�metros gen�ricos que forman parte del comportamiento que tiene $Cplex$ para computar la soluci�n.

En este caso en particular, se tomaron par�metros que se utilizan en el m�todo de Branch and Bound. La prueba entonces, consiste en modificar los valores de dichos par�metros y observar como es el desempe�o del proceso, en tiempo requerido y en cantidad de nodos creados por $Cplex$, hasta obtener la soluci�n del problema.

Los par�metros elegidos son los siguientes:

\begin{itemize}
\item \textbf{CPX\_PARAM\_BRDIR:} Este par�metro sirve para modificar la direcci�n en la cual se har� en branching en cada paso. Los posibles valores para el mismo son los siguientes:
   \begin{itemize}
   \item CPX\_BRDIR\_UP, que indica que el branching siempre debe hacerse por la parte superior.
   \item CPX\_BRDIR\_DOWN, que indica que el branching debe hacerse por la parte inferior.
   \item CPX\_BRDIR\_AUTO, que indica que $Cplex$ decidir� que camino tomar.
   \end{itemize}

\item \textbf{CPX\_PARAM\_LBHEUR:} Indica si las heur�sticas locales en cada branching est�n o no activadas. Los valores para el mismo son: CPX\_ON y CPX\_OFF.

\item \textbf{CPX\_PARAM\_VARSEL:} CPX\_PARAM\_LBHEUR depende de este par�metro, que indica que estrategia usar para seleccionar una variable antes de realizar el branching. Los posibles valores son:
    \begin{itemize}
    \item CPX\_VARSEL\_MININFEAS, que realiza el branch con la variable con inviabilidad m�nima (es decir, la variable fraccionaria mas cercana a alg�n valor entero).
    \item CPX\_VARSEL\_MAXINFEAS, que realiza el braching con la variable mas inviable(es decir la mas lejana a un valor entero)
    \item CPX\_VARSEL\_PSEUDO, la variable es elegida a trav�s de pseudo costos.
    \item CPX\_VARSEL\_PSEUDO, elije la variable a trav�s de la resoluci�n de diferentes sub problemas que permiten saber cuan prometedora es la elecci�n de dicha variable.
    \item CPX\_VARSEL\_PSEUDOREDUCED, selecciona la variable basado en costos.
    \end{itemize}

\item \textbf{CPX\_PARAM\_STRONGITLIM:} Indica el l�mite de iteraciones a realizar en cada una de las variables candidatas para realizar el branching. Los valores que puede tomar son: 0, automatico o un n�mero positivo, que indica las iteraciones fijas a realizar.

\item \textbf{CPX\_PARAM\_ZEROHALFCUTS:} Aqu� se indica si se har�n o no cortes de tipo parte entera hacia abajo para los valores de algunas de las restricciones. Los posibles valores son: -1, desactivado, 0 automatico, 1 normal y 2 agresivo.
\end{itemize}

\subsection{Pruebas realizadas}


\clearpage

\newpage
\chapter{Ejercicio 2}

Para demostrar que las desigualdades son v�lidas, se utilizar� el m�todo del absurdo y demostraci�n por inducci�n. La desigualdad es v�lida para el $CP$ original (sin relajaci�n) si la desigualdad se cumple para todos los puntos dentro de la c�psula convexa de soluciones factibles enteras. Por lo tanto el planteo inicial ser� suponer que el punto no cumple con la nueva restricci�n pero s� con las condiciones originales.

Supongamos que existe un coloreo v�lido para el $CP$ original. Por lo tanto existe $W$ = $(w_0, w_1, ... , w_n)$ y $X = \{x_{vj} | v = 0, ..., n \wedge j = 0, ..., n\}$, donde $v$ son los nodos y $j$ son los colores con los que se pintaron los mismos. Entonces para $X$ vale:

\begin{enumerate}
\item $\displaystyle\sum_{j = 1}^{n} x_{pj} = 1$, $\forall$ $p \in V$

\item $x_{pj} + x_{qj} \le w_j$, $\forall$ $(p,q) \in E$, $j = 1,...,n$

\item $x_{pj} \in \{0,1\}$, $\forall$ $p \in V$, $j = 1,...,n$

\item $w_j \in \{0,1\}$, $\forall$ $j = 1,...,n$

\end{enumerate}

\section{Demostraci�n Desigualdad 1: Clique}

Sea K una Clique del grafo y sea $j_0$ cualquier color tal que $1 \le j_0 \le n$. Supongamos que la soluci�n no cumple con la desigualdad clique, entonces sabemos que vale lo siguiente:

$$\displaystyle\sum_{p \in K} x_{pj_0} > w_{j_0}$$

\begin{itemize}

\item Supongamos que $|K| = 2$. Sean dos nodos cualesquiera $p_1$ y $p_2$ de la Clique, entonces vale que:

$$x_{{p_1}{j_0}} + x_{{p_2}{j_0}} > w_{j_0}$$

Pero por otro lado, como K es una clique, sabemos que existe un eje $({p_1},{p_2})$ de modo que se cumple la condici�n (2) del $CP$ original:

$$x_{{p_1}{j_0}} + x_{{p_2}{j_0}} \le w_{j_0}$$

Pero esto resulta ser absurdo ya que no puede pasar que $A > w_{j_0}$ y que $A \le w_{j_0}$ siendo $A = x_{{p_1}{j_0}} + x_{{p_2}{j_0}}$.$\qed$


\item Supongamos ahora que $|K| > 2$. Sean tres nodos cualesquiera $p_1$, $p_2$ y $p_3$ de la Clique, entonces vale que:

$$x_{{p_1}{j_0}} + x_{{p_2}{j_0}} + x_{{p_3}{j_0}} + ... + x_{{p_n}{j_0}} > w_{j_0}$$

Por otro lado, como K es una clique, sabemos que existen ejes $({p_1},{p_2})$, $({p_2},{p_3})$, $({p_3},{p_1})$ de modo que se cumplen las condiciones (2) del $CP$ original:

$$x_{{p_1}{j_0}} + x_{{p_2}{j_0}} \le w_{j_0}$$

$$x_{{p_2}{j_0}} + x_{{p_3}{j_0}} \le w_{j_0}$$

$$x_{{p_3}{j_0}} + x_{{p_1}{j_0}} \le w_{j_0}$$

\medskip
Entonces vale decir lo siguiente:

$$x_{{p_1}{j_0}} + x_{{p_2}{j_0}} \le w_{j_0} < x_{{p_1}{j_0}} + x_{{p_2}{j_0}} + x_{{p_3}{j_0}} + ... + x_{{p_n}{j_0}}$$

Por lo tanto se deduce lo siguiente:

$$x_{{p_3}{j_0}} + ... + x_{{p_n}{j_0}} > 0$$

Podemos suponer, sin p�rdida de generalidad, que $x_{{p_3}{j_0}} = 1$.

\medskip
Por otro lado tenemos que:

$$x_{{p_2}{j_0}} + x_{{p_3}{j_0}} \le w_{j_0}$$

Por condici�n (2) del $LP$ original. Entonces vale lo siguiente:

$$x_{{p_2}{j_0}} + x_{{p_3}{j_0}} \le w_{j_0} < x_{{p_1}{j_0}} + x_{{p_2}{j_0}} + x_{{p_3}{j_0}} + ... + x_{{p_n}{j_0}}$$

Por lo tanto:

$$x_{{p_1}{j_0}} + x_{{p_4}{j_0}} + ... + x_{{p_n}{j_0}} > 0$$

Nuevamente, sin p�rdida de generalidad suponemos que $x_{{p_1}{j_0}} = 1$. Pero se sab�a por lo anterior que:

$$x_{{p_3}{j_0}} + x_{{p_1}{j_0}} \le w_{j_0}$$

Por lo tanto, se llega a que:

$$w_{j_0} \geq 2$$

Lo cual resulta absurdo, pues por (4) $w_j \in \{0, 1\} \forall 1 \le j \le n$. $\qed$

\end{itemize}

Dado que no hay m�s alternativas, la suposici�n de que la desigualdad no se cumple es falsa. $\qed$

\section{Demostraci�n Desigualdad 2: Agujero de Longitud Impar}

Sea $C_{2k+1}$ el conjunto de v�rtices dentro de un agujero de longitud $2k+1$, con $k >= 2$. Supongamos que la soluci�n no cumple con la desigualdad agujero impar. Por lo tanto podemos decir que $\exists k >= 2$ tal que:

$$
\sum_{p \in C_{2k+1}} x_{p{j_0}} > kw_{j_0}
$$

Con $j_0$ cualquier color tal que $1 \le j_0 \le n$.

\begin{itemize}
\item Supongamos que $w_{j_0} = 0$. Para este caso la desigualdad quedar�a de la siguiente forma:

$$
\sum_{p \in C_{2k+1}} x_{p{j_0}} > 0
$$
      
Como se puede observar, la restricci�n se cumple si existe alg�n nodo $p$ del agujero $C_{2k+1}$ tal que $x_{p{j_0}} = 1$. Dado que no hay nodos aislados, podemos tomar un eje $(p,p')$ de modo que se cumple la condici�n (2) del $CP$ original:

$$x_{{p}{j_0}} + x_{{p'}{j_0}} \le w_{j_0}$$

O sea que

$$x_{{p'}{j_0}} + 1 \le w_{j_0}$$

Pero por (3) $x_{{p'}{j_0}} \in \{0,1\}$, y $w_{j_0} = 0$, o sea que $x_{{p'}{j_0}} + 1 \le 0$ lo cual es absurdo. $\qed$

\item Supongamos ahora que $w_{j_0} = 1$. Para este caso la desigualdad quedar�a de la siguiente forma:

$$
\sum_{p \in C_{2k+1}} x_{p{j_0}} > k
$$
      
La restricci�n entonces se cumple si hay almenos $k+1$ nodos del agujero pintados con el color $j_0$.

Para un circuito de longitud $2k+1$, la m�xima cantidad de nodos pintados del mismo color para que su coloreo sea v�lido es $k$. Como los nodos del agujero conforman un circuito de longitud impar, si pintamos $k+1$ o m�s nodos del mismo color, este coloreo ser�a inv�lido para el mismo circuito, por lo tanto hay almenos dos nodos que son adyacentes entre s� y est�n pintados del mismo color. Por consiguiente el coloreo es inv�lido y la suposici�n de que $w_{j_0} = 1$ es absurda. $\qed$

\end{itemize}

Dado que no hay m�s alternativas, la suposici�n de que la desigualdad no se cumple es falsa. $\qed$


\section{Heur�stica de Separaci�n: Clique}

Se implement� un callback llamado $cortes$ que se lo setea a $CPLEX$ como heur�stica de separaci�n. Este callback se ejecuta juesto luego de que $CPLEX$ haya encontrado la soluci�n para un nodo del �rbol de branching, y llama a nuestra funci�n de separaci�n por cortes clique, llamada $corteClique$.

La idea de este m�todo es bastante simple, y se procede a realizar la descripci�n de sus pasos a continuaci�n:

\begin{enumerate}
\item Se toma la soluci�n $X$ que encontr� $CPLEX$ para el nodo actual del �rbol de branching. Este vector tiene una cantidad determinada columnas, las cuales se corresponden con las variables $w_j$ de colores y $x_pj$ de nodos/colores del grafo. De todas estas columnas nos quedamos solamente con aquellas que se correspondan con los $x_pj$ tal que su valor $k$ sea $0 \le k < 1$. Recordar que estos valores son de punto flotante debido a la relajaci�n.

\item Dado que buscamos la clique de mayor valor de sumatoria del valor de sus variables asociadas, recorremos este vector de mayor a menor intentando agregar un nuevo nodo en cada paso. Notar que este es un vector de variables del modelo, por lo tanto para testear si se puede agregar o no un nuevo nodo a la clique en construcci�n, dada la columna $x_pj$, averiguamos el $p$ y el $j$ confiando fuertemente en el formato del archivo $.lp$ de entrada, ya que estos c�lculos dependen del orden de las columnas en la secci�n $Binary$. Si $j$ es del mismo color que el primer elemento del vector ordenado (que siempre es agregado como primer nodo) y si $p$ es vecino de todos los nodos de la clique actual, se procede a agregar el nodo y se guarda tambi�n la columna relacionada.

\item Luego de agregar un nuevo nodo verificamos si la desigualdad

$$\displaystyle\sum_{p \in K} x_{pj_0} > w_{j_0} + \epsilon$$

se cumple. Esto quiere decir que la desigualdad clique es violada y por lo tanto podemos agregar un corte (gracias a lo demostrado anteriormente) y la heur�stica termina. Utilizamos un margen de error $\epsilon = 0.000001$.
\end{enumerate}

A continuaci�n se agrega el pseudoc�digo para mostrar mejor la idea que se explico anteriormente.

\begin{algorithm}[H]
\caption{Corte Clique}
%\label{alg:algoritmo2_1}
\begin{algorithmic}[1]
\PARAMS{Datos de $CPLEX$}
\STATE Ordenar los x\'s de acuerdo a su valor y que est�n entre 0 (inclusive) y 1 (exclusive).
\STATE Obtener el nodo asociado a la columna $X_m$ con mayor valor.
\STATE Agregar $X_m$ a la clique $K$
\FOR{Cada columna $X_i$ que siguen en el ordenamiento y que tenga el mismo color que $X_m$}
	\IF{El nodo relacionado con la columna $X_i$ es adyacente a todos los nodos de $K$}
		\STATE agregar el nodo a $K$
		\IF{la sumatoria de valores de $K$ es mayor que el valor de $w_{j0} + \epsilon$, donde j0 es el color de $X_m$}
			\STATE Generar la desigualdad basada en $K$, agregarla al problema y salir del ciclo.
		\ENDIF
	\ENDIF
\ENDFOR
\end{algorithmic}
\end{algorithm}

La idea de esta heur�stica fue extra�da del paper ``A branch-and-cut algorithm for graph coloring'' de Isabel M�ndez D�az y Paula Zabala, como se puede ver en las referencias (secci�n \ref{cliquecutalgorithm}).


\section{Heur�stica de Separaci�n: Agujero Impar}

El m�todo que realiza los cortes de tipo agujero impar es $corteCicloImpar$, que es llamado por $cortes$, el cual a su vez se llama cada vez que $CPLEX$ intenta hacer un corte. La idea del funcionamiento de este m�todo es simple. Se parte de la misma base que la heur�stica de clique descripta anteriormente, es decir partiendo de la base de que los valores relacionados a cada nodo se encuentran ordenados de mayor a menor, para maximizar la probabilidad en este caso de encontrar un ciclo que viole la desigualdad agujero impar. A continuaci�n se describen los pasos mas importantes del algoritmo.

\begin{enumerate}
\item Una vez recibidos los datos que son pasados por $CPLEX$ y con los nodos ordenados de mayor a menor, lo que se hace es partir del nodo con mayor valor y realizar un BFS, de manera de construir un �rbol en base al grafo relacionado al problema actual. El m�todo hace varias iteraciones partiendo siempre desde distinto nodo, de mayor a menor valor de columna asociada. Tiene un l�mite dado para la cantidad de iteraciones y para cada iteraci�n, otro l�mite dado para la cantidad de ciclos verificados.
\item A medida que se hace el BFS y a medida que va creciendo el �rbol, se verifica si existen ciclos impares entre cada par de nodos $h_1$ y $h_2$ que sean hojas en ese paso y sean adyacentes entre s�. Esto no se hace para todo par de hojas $h_1$ y $h_2$, sino solamente para aquellas que vengan de diferente hijo de la ra�z. Es decir, dos hojas $h_1$ y $h_2$ vienen de diferentes hijos de la ra�z $r_1$ y $r_2$ si, al eliminar la ra�z del �rbol de BFS actual, existe un camino desde $r_1$ a $h_1$ y otro desde $r_2$ a $h_2$, con $r_1 \neq r_2$.
\item Un vez que se sabe que existe un ciclo impar se llama a otro m�todo: $tryToAddOddCycleCut$. Este se encarga de reconstruir el ciclo encontrado y verificar si la desigualdad de agujero impar no se cumple para cada uno de los colores disponibles. Si no se cumple, entonces se agrega una restricci�n al problema (notar que en una llamada a $tryToAddOddCycleCut$ se puede agregar m�s de 1 restricci�n al problema, si es que para 2 o mas colores se encuentra que la desigualdad se viola), sino se contin�a repitiendo el proceso de BFS.
Cabe aclarar adem�s, que si bien para cada ciclo encontrado, pueden agregarse 0, 1 o m�s restricciones nuevas al problema, se pueden agregar m�s restricciones a medida que se sigue recorriendo el grafo en orden BFS, ya que cuando se encuentra un ciclo y este genera alguna restricci�n nueva al problema, el proceso no se detiene hasta tanto se haya recorrido toda la componente conexa o se haya llegado al l�mite de b�squeda de ciclos.
\end{enumerate}

Para explicar mejor el algoritmo, veamos su pseudoc�digo:

\begin{algorithm}[H]
\caption{Corte Agujero Impar}
%\label{alg:algoritmo2_1}
\begin{algorithmic}[1]
\PARAMS{Datos de $CPLEX$}
\STATE Ordenar los x\'s de acuerdo a su valor y que est�n entre 0 y 1.
\STATE Obtener el nodo asociado $r$ a la columna $X_m$ con mayor valor.
\STATE \COMMENT{Se comienza a realizar el proceso de BFS.}
\STATE Tomar el nodo $r$ como ra�z principal para formar el �rbol $A_{BFS}$, marcar su nivel en $0$, su ra�z $n_m$ y su padre $-1$.
\FOR{Cada nodo $r_{ady}$ adyacente a $r$}
	\STATE $Nivel(r_{ady}) \leftarrow 2$
	\STATE $Raiz(r_{ady}) \leftarrow r_{ady}$
	\STATE $Padre(r_{ady}) \leftarrow r$
\ENDFOR
\FOR{Cada nodo $n_i$ que sigue en el recorrido BFS}
	\IF{A�n no se alcanz� el l�mite de ciclos verificados}
		\FOR{Cada nodo $n_k$ en $Vecinos(n_i)$}
			\IF{$Nivel(n_k) > -1$}
				\STATE $tamanoCiclo \leftarrow Nivel(n_i) + Nivel(n_k) + 1$
				\IF{$n_k$ no es el padre de $n_i$ y tamanoCiclo es impar, mayor a 4 y $Raiz(n_k) \neq Raiz(n_i)$}
						\STATE $tryToAddOddCycleCut(n_k, n_i, Padre, r)$
				\ENDIF
			\ELSE
				\STATE $Nivel(n_k) \leftarrow Nivel(n_i) + 1$
				\STATE $Raiz(n_k) \leftarrow Raiz(n_i)$
				\STATE $Padre(n_k) \leftarrow n_i$
			\ENDIF
		\ENDFOR
	\ENDIF
\ENDFOR
\end{algorithmic}
\end{algorithm}

La idea del m�todo $tryToAddOddCycleCut$ es simple y no necesita demasiado detalle. En este se aprobecha la informaci�n constru�da de los padres de cada nodo para reconstruir el ciclo. A su vez se va calculando la sumatoria de los valores correspondientes a las columnas $x_{pj_0}$ tal que $p \in OddCycle$ y todas tienen el mismo $j_0$. Luego este valor se lo compara con $w_{j_0}$ de la siguiente manera:

$$\sum_{p \in C_{2k+1}} x_{p{j_0}} > kw_{j_0}$$

siendo k el tama�o del ciclo. Si esto se cumple quiere decir que la desigualdad $Odd-Hole$ es violada, por lo tanto agregamos una restricci�n al problema. Adem�s, como dijimos, se prueba esta desigualdad para todo color $j_0$.

La idea del algoritmo fue extra�da del paper ``Solving Airline Crew Scheduling Problems by Branch-and-Cut'' de Karla L. Hoffman y Manfred Padberg, como se puede ver en las referencias (secci�n \ref{oddcyclecutalgorithm}).
\clearpage

%\newpage
%\chapter{Ejercicio 3}

Se realiz� la implementaci�n del algoritmo de Branch-and-Cut pedido en el enunciado, incorporando el algoritmo de Planos de Corte en el Branch and Bound de $CPLEX$ como se muestra en el diagrama de flujo a continuaci�n.

\section{Flujo del algoritmo de Branch-and-Cut para el ejercicio 3}

\begin{figure}[H]
\centering
\includegraphics[scale=0.2]{figuras/flow2.png}
\caption{Flujo del algoritmo}
\end{figure}
%\clearpage

%\newpage
%\section{Ejercicio 4}

\subsection{Tipos de tests y casos}

En esta secci�n se har�n comparaciones entre los algoritmos de las secciones anteriores con un algoritmo que corre $CPLEX$ standard con los par�metros por defecto. Adem�s, se analizar� c�mo aportan nuestras heur�sticas y algoritmos de separaci�n a $CPLEX$ por defecto.

Los an�lisis y las comparaciones se har�n en base a una serie de tipos de tests que tienen diferentes configuraciones entre los cortes a usar, las heur�sticas iniciales y diferentes par�metros que definen el comportamiento del Branching. Los tipos de test, son los siguientes:

\begin{enumerate}
\item Tipo 1: Se setean las heur�sticas iniciales propias, sin activar las heur�sticas iniciales de $CPLEX$. Se desactivan todos los tipos de cortes. Se setean los par�metros de Branching seg�n el tipo 7 de los tipos del ejercicio n�mero 1.

\item Tipo 2: Se setean las heur�sticas iniciales propias, sin activar las de $CPLEX$. Se activan los cortes de $CPLEX$ (sin los cortes Clique y Agujero Impar implementados). Se setean los par�metros de Branching seg�n el tipo 7 de los tipos del ejercicio n�mero 1.

\item Tipo 3: Se setean las heur�sticas iniciales propias, sin activar las de $CPLEX$. Se activan los cortes de $CPLEX$ y los cortes Clique y Agujero Impar implementados. Se setean los par�metros de Branching seg�n el tipo 7 de los tipos del ejercicio n�mero 1.

\item Tipo 4: Se setean las heur�sticas iniciales propias, sin activar las de $CPLEX$. Se activan los cortes de $CPLEX$ y los cortes Clique y Agujero Impar implementados. Se setean los par�metros de Branching de forma autom�tica, dejando que $CPLEX$ elija el apropiado seg�n el caso.

\item Tipo 5: Solo $CPLEX$. Se setean las heur�sticas iniciales y cortes de $CPLEX$. Se setean los par�metros de Branching de forma autom�tica, dejando que $CPLEX$ elija el apropiado seg�n el caso.

\end{enumerate}

Con estos tipos de tests, se espera ver como se comportan las heur�sticas implementadas versus las heur�sticas iniciales de $CPLEX$, el rendimiento en tiempo y cantidad de nodos entre $CPLEX$ autom�tico versus las variantes que se ten�an en secciones anteriores.

A continuaci�n se describir�n cuales son las instancias con las cuales se har�n las pruebas para cada tipo:

\begin{enumerate}
\item test 1: La primera de las instancias es el grafo myciel4, que es un grafo libre de tri�ngulos y con 23 nodos.

\item test 2: Grafo generado aleatoriamente, con una cantidad de 42 nodos y una densidad de 25\%.

\item test 3: Grafo generado aleatoriamente, con una cantidad de 42 nodos y una densidad de 50\%.

\item test 4: Grafo generado aleatoriamente, con una cantidad de 42 nodos y una densidad de 75\%.

\item test 5: Grafo generado aleatoriamente, con una cantidad de 45 nodos y una densidad de 25\%.

\item test 6: Grafo generado aleatoriamente, con una cantidad de 45 nodos y una densidad de 50\%.

\item test 7: Grafo generado aleatoriamente, con una cantidad de 45 nodos y una densidad de 75\%.

\item test 8: La �ltima instancia es el grafo myciel5, que es un grafo libre de tri�ngulos y con 47 nodos.

\end{enumerate}

\subsection{Resultados y comparaci�n final}

\begin{figure}[H]
\centering
\includegraphics[scale=0.7]{../tests/resultados/resultados5/graficoTiemposEj4Tp.png}
\caption{Tiempo en segundos para cada test}
\end{figure}

En este gr�fico se puede observar que $CPLEX$ configurado en todo autom�tico generalmente prevalece en tiempos. Esto puede ser debido a que los cortes implementados no son del todo performantes o que $CPLEX$ se configura bien autom�ticamente. De todos modos los tiempos de ejecuci�n se pueden llegar a reducir si probamos con otras configuraciones de los par�metros de $CPLEX$. Algo que llama la atenci�n es que en el test n�mero cinco, $CPLEX$ es notablemente mejor que los dem�s resolvedores, evidentemente hay muchas cosas que no se tienen en cuenta a la hora de elegir mejores camimos en el branching, o a la hora de realizar cortes de una naturaleza distinta a las de clique o agujero impar. Cabe aclarar tambi�n, que como $CPLEX$ se encuentra configurado con su preproceso activado para el tipo 5, puede pasar que se est�n utilizando otras heur�sticas de reducci�n del problema, diferentes a las implementadas por nosotros.
Es interesante destacar que, al contrario de lo que cre�amos, el tipo 1 (Branch and Bound) logr� una no tan mala performance en comparaci�n con todos los dem�s tipos.

\begin{figure}[H]
\centering
\includegraphics[scale=0.7]{../tests/resultados/resultados5/graficoNodosEj4Tp.png}
\caption{Nodos explorados del Branching Tree para cada test}
\end{figure}

En este gr�fico nuevamente se puede observar que $CPLEX$ configurado en todo autom�tico generalmente recorre menos nodos. Esto puede ser debido a que los cortes implementados no son del todo eficientes o que $CPLEX$ se configura bien autom�ticamente. Sin embargo puede notarse diferencias entre aquellos tests que utilizan los cortes implementados y aquellos que no, pues por lo general estos �ltimos recorren mayor cantidad de nodos que los primeros.

\begin{figure}[H]
\centering
\includegraphics[scale=0.7]{../tests/resultados/resultados5/graficoCortesCliqueEj4Tp.png}
\caption{Cortes Clique para cada test}
\end{figure}

En este gr�fico se compara la cantidad de cortes Clique (de los implementados) hechos en cada caso de test que los tenga activados. Por razones obvias no se incluyen los otros tipos de test en la comparaci�n. Se puede confirmar que se logra hacer un gran aporte a $CPLEX$ en cuanto a la cantidad de cortes en cada nodo del Branching Tree. Es por esto que anteriormente en el gr�fico de la cantidad de nodos explorados ve�amos una diferencia entre aquellos tipos de test que usaban los cortes y aquellos que no. Notemos que en el primer test no se hacen cortes Clique. Esto es correcto ya que este grafo la clique m�xima es de 2, por lo tanto las desigualdades clique verificadas son las mismas que las restricciones por ejes, que siempre se cumplen.


\begin{figure}[H]
\centering
\includegraphics[scale=0.7]{../tests/resultados/resultados5/graficoCortesOddHoleEj4Tp.png}
\caption{Cortes Odd-Hole para cada test}
\end{figure}

En este gr�fico se compara la cantidad de cortes Odd-Hole (de los implementados) hechos en cada caso de test que los tenga activados. Por razones obvias no se incluyen los otros tipos de test en la comparaci�n. Como se ve este tipo de cortes no hace gran aporte a $CPLEX$, pero pudimos confirmar que el mismo es muy eficiente para por lo menos una cierta clase de grafos: aquellos tal que su clique m�xima es de 2 (dos) nodos. Se ejecutaron los mismos tipos de test para el test 8 (myciel5) y se logra una excelente performance:
\[
\begin{array}{cccc}
	tiempo(segundos)	&	Odd-Hole	&	Clique	&	nodos\\
	\hline
	19.782	&	254 & 0 & 772 \\
	23.953	&	221 & 0 & 849 \\
\end{array}
\]

Los test de tipo 1, 2 y 5 que son los que tienen los cortes desactivados tardan un tiempo significativamente mayor (m�s de 20 minutos).

Hubieron much�simos tests generados al azar (de hasta 80 nodos) que no tardaron en resolverse gracias a que las heur�sticas iniciales daban igual cota inferior que superior, por lo que la funci�n objetivo quedaba vac�a. Sin embargo estos tests no fueron inclu�dos en este informe debido a que no aportaban informaci�n alguna para este an�lisis.
%\clearpage

\label{LastPage}
\end{document}

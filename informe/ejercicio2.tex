\section{Ejercicio2}

Para demostrar que las desigualdades son v�lidas, se utilizar� el m�todo del absurdo. La desigualdad es v�lida para el $CP$ original (sin relajaci�n) si la desigualdad se cumple para todos los puntos dentro de la c�psula convexa de soluciones factibles enteras. Por lo tanto el planteo inicial ser� suponer que el punto no cumple con la nueva restricci�n pero s� con las condiciones originales.

\subsection{Desigualdad 1: Clique}

Supongamos que existe un coloreo v�lido para el $CP$ original. Por lo tanto existe $W$ = $(w_0, w_1, ... , w_n)$ y $X = \{x_{vj} | v = 0, ..., n \wedge j = 0, ..., n\}$, donde $v$ son los nodos y $j$ son los colores con los que se pintaron los mismos. Entonces para $X$ vale:

\begin{enumerate}
\item $\displaystyle\sum_{j = 1}^{n} x_{pj} = 1$, $\forall$ $p \in V$

\item $x_{pj} + x_{qj} \le w_j$, $\forall$ $(p,q) \in E$, $j = 1,...,n$

\item $x_{pj} \in \{0,1\}$, $\forall$ $p \in V$, $j = 1,...,n$

\item $w_j \in \{0,1\}$, $\forall$ $j = 1,...,n$

\end{enumerate}


Supongamos adem�s que dicha soluci�n no cumple con la desigualdad clique, entonces sabemos que vale lo siguiente:

$$\displaystyle\sum_{p \in K} x_{pj_0} > w_{j_0}$$

\begin{itemize}

\item Supongamos que $|K| = 2$, entonces vale que:

$$x_{{p_1}{j_0}} + x_{{p_2}{j_0}} > w_{j_0}$$

Pero por otro lado, como K es una clique, sabemos que existe un eje $({p_1},{p_2})$ de modo que se cumple la condici�n (2) del $CP$ original:

$$x_{{p_1}{j_0}} + x_{{p_2}{j_0}} <= w_{j_0}$$

Pero esto resulta ser absurdo ya que no puede pasar que $A > w_{j_0}$ y que $A <= w_{j_0}$ siendo $A = x_{{p_1}{j_0}} + x_{{p_2}{j_0}}$.$\qed$


\item Supongamos ahora que $|K| > 2$, entonces:

$$x_{{p_1}{j_0}} + x_{{p_2}{j_0}} + x_{{p_3}{j_0}} + ... + x_{{p_n}{j_0}} > w_{j_0}$$

Por otro lado, como K es una clique de al menos tres nodos, sabemos que existen ejes $({p_1},{p_2})$, $({p_2},{p_3})$, $({p_3},{p_1})$ de modo que se cumplen las condiciones (2) del $CP$ original:

$$x_{{p_1}{j_0}} + x_{{p_2}{j_0}} \le w_{j_0}$$

$$x_{{p_2}{j_0}} + x_{{p_3}{j_0}} \le w_{j_0}$$

$$x_{{p_3}{j_0}} + x_{{p_1}{j_0}} \le w_{j_0}$$

\medskip
Entonces vale decir lo siguiente:

$$x_{{p_1}{j_0}} + x_{{p_2}{j_0}} < x_{{p_1}{j_0}} + x_{{p_2}{j_0}} + x_{{p_3}{j_0}} + ... + x_{{p_n}{j_0}}$$

Por lo tanto podemos llegar a lo siguiente:

$$x_{{p_3}{j_0}} + ... + x_{{p_n}{j_0}} > 0$$

Sin p�rdida de generalidad suponemos que $x_{{p_3}{j_0}} = 1$.

\medskip
Por otro lado tenemos que:

$$x_{{p_2}{j_0}} + x_{{p_3}{j_0}} \le w_{j_0}$$

Por condici�n (2) del $LP$ original. Entonces vale lo siguiente:

$$x_{{p_2}{j_0}} + x_{{p_3}{j_0}} < x_{{p_1}{j_0}} + x_{{p_2}{j_0}} + x_{{p_3}{j_0}} + ... + x_{{p_n}{j_0}}$$

Por lo tanto:

$$x_{{p_1}{j_0}} + x_{{p_4}{j_0}} + ... + x_{{p_n}{j_0}} > 0$$

Nuevamente, sin p�rdida de generalidad suponemos que $x_{{p_1}{j_0}} = 1$. Pero se sab�a por lo anterior que:

$$x_{{p_3}{j_0}} + x_{{p_1}{j_0}} \le w_{j_0}$$

Por este motivo, se llega a lo siguiente:

$$w_{j_0} \geq 2$$

Lo cual resuulta absurdo, pues $w_{j_0} \in \{0, 1\}$. $\qed$

\end{itemize}


\subsection{Desigualdad 2: Agujero de Longitud Impar}

Supongamos que existe un coloreo v�lido para el $CP$ original. Por lo tanto existe $W$ = $(W_1, W_2, ... , W_n)$ una cantidad de colores para pintar el grafo y $X$ = $(X_{1j_{x1}}, ... , X_{nj_{xn}})$ el coloreo correspondiente, donde $j_{x1}, ... , j_{xn}$ son los colores con los que se pintaron cada uno de los nodos $X_1, ... , X_n$.

Supongamos adem�s que dicha soluci�n no cumple con la desigualdad agujero impar. Por lo tanto podemos decir que $\exists k >= 2$ tal que:

$$
\sum_{p \in C_{2k+1}} X_{p,j_0} > kW_{j_0}
$$

Donde $C_{2K+1}$ es el conjunto de v�rtices dentro del agujero de longitud impar $2K+1$ y donde $j_0 = (1, ... , n)$. Como se puede observar dicha desigualdad, se encuentra dominada por el valor que tome $W_{j_0}$, es decir, si el color $j_0$ se usa o no.

Si llega a ser el caso en el que $W_{j_0} = 0$, la desigualdad nos quedar�a de la siguiente forma:

\begin{itemize}
\item $$
      \sum_{p \in C_{2K+1}} X_{p,j_0} > 0
      $$
      
      Como se puede observar, la restricci�n se cumple si existe alg�n $p$ tal que $X_{o,j_0} = 1$. Pero como se puede ver, esto pasa si y solo si hay un nodo pintado con el color $j_0$, pero esto no puede ser, ya que se hab�a dicho que $W_{j_0} = 0$, que indica que el color $j_0$ no se usa. Por lo tanto llegamos a un absurdo.$\qed$
\end{itemize}

Si llega a ser el caso en el que $W_{j_0} = 1$, la desigualdad quedar�a de la siguiente forma:

\begin{itemize}
\item$$
      \sum_{p \in C_{2K+1}} X_{p,j_0} > K
      $$
      
      La restricci�n entonces, se cumple si pasa que hay $K+1$ o mas nodos pintados con el color $j_0$ dentro del agujero impar. Quiere decir que en un agujero de $2K+1$ nodos, est�n pintados $k+1$ o mas del mismo color, y esto pasa si y solo si hay al menos dos nodos adyacentes $X_{p, j_0}$ y $X_{q, j_0}$ (pintados del mismo color).
      
      Pero por una de las restricciones del $LP$ original, tiene que valer que:
      $$
      X_{p, j_0} + X_{q, j_0} \le W_{j_0} = 1 \leftrightarrow 2 \le 1
      $$
      
      Lo cual resulta absurdo, ya que la soluci�n era v�lida para el $CP$ original.$\qed$
\end{itemize}
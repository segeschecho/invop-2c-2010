 %%	SECCION documentclass																									 %%	
%%---------------------------------------------------------------------------%%
\documentclass[a4paper]{report}

%%---------------------------------------------------------------------------%%
%%	SECCION usepackage																											 %%	
%%---------------------------------------------------------------------------%%
\usepackage{amsmath, amsthm}
\usepackage[spanish,activeacute]{babel}
\usepackage{caratula}
\usepackage{a4wide}
\usepackage{hyperref}
\usepackage{fancyhdr}
\usepackage{graphicx} % Para el logo magico!
\usepackage{amssymb}
\usepackage{amsmath}
\usepackage{float}
\usepackage[latin1]{inputenc}
%\usepackage [T1]{fontenc}
\usepackage[dvipsnames,usenames]{color}
\usepackage{amsfonts}
\usepackage{ulem}
%\usepackage{highlight}
\usepackage{fancybox}
%\usepackage{marvosym}
\usepackage{color}
\usepackage{lastpage}
\usepackage{lscape}
\usepackage{tabularx}
\usepackage{algorithmic}
\usepackage{algorithm}
\usepackage{subfigure}

%%---------------------------------------------------------------------------%%
%%	SECCION opciones																												 %%	
%%---------------------------------------------------------------------------%%
\parskip    = 11 pt
\headheight	= 13.1pt
\pagestyle	{fancy}
\definecolor{orange}{rgb}{1,0.5,0}

\addtolength{\headwidth}{1.0in}

\addtolength{\oddsidemargin}{-0.5in}
\addtolength{\textwidth}{1.0in}
\addtolength{\topmargin}{-0.5in}
\addtolength{\textheight}{0.7in}

%%---------------------------------------------------------------------------%%
%%	SECCION document	 %%	
%%---------------------------------------------------------------------------%%
\begin{document}
\renewcommand{\chaptername}{Parte }
\renewcommand{\algorithmicrequire}{\textcolor{blue}{\textbf{Requiere:}}}
\renewcommand{\algorithmicensure}{\textbf{Asegura:}}
\renewcommand{\algorithmicend}{\textbf{Fin}}
\renewcommand{\algorithmicif}{\textcolor{blue}{\textbf{Si}}}
\renewcommand{\algorithmicthen}{\textcolor{blue}{\textbf{entonces}}}
\renewcommand{\algorithmicelse}{\textcolor{red}{\textbf{Si no}}}
\renewcommand{\algorithmicelsif }{\textcolor{blue}{\textbf{Si no y}}}
\renewcommand{\algorithmicendif}{\textcolor{blue}{\textbf{Fin si}}}
\renewcommand{\algorithmicfor}{\textcolor{ForestGreen}{\textbf{Para}}}
\renewcommand{\algorithmicendfor}{\textcolor{ForestGreen}{\textbf{Fin para}}}
\renewcommand{\algorithmicwhile}{\textcolor{ForestGreen}{\textbf{Mientras}}}
\renewcommand{\algorithmicendwhile}{\textcolor{ForestGreen}{\textbf{Fin mientras}}}
\renewcommand{\algorithmicdo}{\textcolor{ForestGreen}{\textbf{hacer}}}
\renewcommand{\algorithmicreturn}{\textbf{Devolver}}
\floatname{algorithm}{Algoritmo}

%%---- Caratula -------------------------------------------------------------%%
\materia{Investigaci�n Operativa (2do cuatrimestre de 2010)}
\titulo{Trabajo Pr�ctico}

\integrante{Gonzalez, Emiliano}{426/06}{xjesse\_jamesx@hotmail.com}
\integrante{Gonzalez, Sergio}{481/06}{gonzalezsergio2003@yahoo.com.ar}
\resumen{
En el siguiente documento, se mostrar�n diferentes pruebas de performance utilizando la herramienta CPLEX. Las pruebas consisten en la modificaci�n de diferentes t�cnicas de brancheo del m�todo Branch and Bound y del m�todo Branch and Cut, esta �ltima con una parte implementada especialmente para resolver el problema de separaci�n.}

% TOC, usa estilos locos
\maketitle
\pagestyle{empty}
{
\fancypagestyle{plain}
    {
    \fancyhead{}
    \fancyfoot{}
    \renewcommand{\headrulewidth}{0.0pt}
    } % clear header and footer of plain page because of ToC
\tableofcontents
}

\newpage
% arreglos los estilos para el resto del documento, y
% reseteo los numeros de pagina para que queden bien
\pagenumbering{arabic}
\fancypagestyle{plain} {
    \fancyhead[LO]{Gonz�lez, Gonz�lez}
    \fancyhead[C]{}
    \fancyhead[RO]{P\'agina \thepage\ de \pageref{LastPage}}
    \fancyfoot{}
    \renewcommand{\headrulewidth}{0.4pt}
}
\pagestyle{plain}

\newpage
\section{Ejercicio1}

El objetivo de este ejercicio consiste en poner a prueba el problema de coloreo, variando algunos de los par�metros que $Cplex$ utiliza para resolver el problema. Los mismos son par�metros gen�ricos que forman parte del comportamiento que tiene $Cplex$ para computar la soluci�n.

En este caso en particular, se tomaron par�metros que se utilizan en el m�todo de Branch and Bound. La prueba entonces, consiste en modificar los valores de dichos par�metros y observar como es el desempe�o del proceso, en tiempo requerido y en cantidad de nodos creados por $Cplex$, hasta obtener la soluci�n del problema.

Los par�metros elegidos son los siguientes:

\begin{itemize}
\item \textbf{CPX\_PARAM\_BRDIR:} Este par�metro sirve para modificar la direcci�n en la cual se har� en branching en cada paso. Los posibles valores para el mismo son los siguientes:
   \begin{itemize}
   \item CPX\_BRDIR\_UP, que indica que el branching siempre debe hacerse por la parte superior.
   \item CPX\_BRDIR\_DOWN, que indica que el branching debe hacerse por la parte inferior.
   \item CPX\_BRDIR\_AUTO, que indica que $Cplex$ decidir� que camino tomar.
   \end{itemize}

\item \textbf{CPX\_PARAM\_LBHEUR:} Indica si las heur�sticas locales en cada branching est�n o no activadas. Los valores para el mismo son: CPX\_ON y CPX\_OFF.

\item \textbf{CPX\_PARAM\_VARSEL:} CPX\_PARAM\_LBHEUR depende de este par�metro, que indica que estrategia usar para seleccionar una variable antes de realizar el branching. Los posibles valores son:
    \begin{itemize}
    \item CPX\_VARSEL\_MININFEAS, que realiza el branch con la variable con inviabilidad m�nima (es decir, la variable fraccionaria mas cercana a alg�n valor entero).
    \item CPX\_VARSEL\_MAXINFEAS, que realiza el braching con la variable mas inviable(es decir la mas lejana a un valor entero)
    \item CPX\_VARSEL\_PSEUDO, la variable es elegida a trav�s de pseudo costos.
    \item CPX\_VARSEL\_PSEUDO, elije la variable a trav�s de la resoluci�n de diferentes sub problemas que permiten saber cuan prometedora es la elecci�n de dicha variable.
    \item CPX\_VARSEL\_PSEUDOREDUCED, selecciona la variable basado en costos.
    \end{itemize}

\item \textbf{CPX\_PARAM\_STRONGITLIM:} Indica el l�mite de iteraciones a realizar en cada una de las variables candidatas para realizar el branching. Los valores que puede tomar son: 0, automatico o un n�mero positivo, que indica las iteraciones fijas a realizar.

\item \textbf{CPX\_PARAM\_ZEROHALFCUTS:} Aqu� se indica si se har�n o no cortes de tipo parte entera hacia abajo para los valores de algunas de las restricciones. Los posibles valores son: -1, desactivado, 0 automatico, 1 normal y 2 agresivo.
\end{itemize}

\subsection{Pruebas realizadas}


\clearpage

\newpage
\section{Ejercicio 2}

Para demostrar que las desigualdades son v�lidas, se utilizar� el m�todo del absurdo. La desigualdad es v�lida para el $CP$ original (sin relajaci�n) si la desigualdad se cumple para todos los puntos dentro de la c�psula convexa de soluciones factibles enteras. Por lo tanto el planteo inicial ser� suponer que el punto no cumple con la nueva restricci�n pero s� con las condiciones originales.

Supongamos que existe un coloreo v�lido para el $CP$ original. Por lo tanto existe $W$ = $(w_0, w_1, ... , w_n)$ y $X = \{x_{vj} | v = 0, ..., n \wedge j = 0, ..., n\}$, donde $v$ son los nodos y $j$ son los colores con los que se pintaron los mismos. Entonces para $X$ vale:

\begin{enumerate}
\item $\displaystyle\sum_{j = 1}^{n} x_{pj} = 1$, $\forall$ $p \in V$

\item $x_{pj} + x_{qj} \le w_j$, $\forall$ $(p,q) \in E$, $j = 1,...,n$

\item $x_{pj} \in \{0,1\}$, $\forall$ $p \in V$, $j = 1,...,n$

\item $w_j \in \{0,1\}$, $\forall$ $j = 1,...,n$

\end{enumerate}

\subsection{Desigualdad 1: Clique}

Sea K una Clique del grafo y sea $j_0$ cualquier color tal que $1 \le j_0 \le n$. Supongamos que la soluci�n no cumple con la desigualdad clique, entonces sabemos que vale lo siguiente:

$$\displaystyle\sum_{p \in K} x_{pj_0} > w_{j_0}$$

\begin{itemize}

\item Supongamos que $|K| = 2$. Sean dos nodos cualesquiera $p_1$ y $p_2$ de la Clique, entonces vale que:

$$x_{{p_1}{j_0}} + x_{{p_2}{j_0}} > w_{j_0}$$

Pero por otro lado, como K es una clique, sabemos que existe un eje $({p_1},{p_2})$ de modo que se cumple la condici�n (2) del $CP$ original:

$$x_{{p_1}{j_0}} + x_{{p_2}{j_0}} \le w_{j_0}$$

Pero esto resulta ser absurdo ya que no puede pasar que $A > w_{j_0}$ y que $A \le w_{j_0}$ siendo $A = x_{{p_1}{j_0}} + x_{{p_2}{j_0}}$.$\qed$


\item Supongamos ahora que $|K| > 2$. Sean tres nodos cualesquiera $p_1$, $p_2$ y $p_3$ de la Clique, entonces vale que:

$$x_{{p_1}{j_0}} + x_{{p_2}{j_0}} + x_{{p_3}{j_0}} + ... + x_{{p_n}{j_0}} > w_{j_0}$$

Por otro lado, como K es una clique, sabemos que existen ejes $({p_1},{p_2})$, $({p_2},{p_3})$, $({p_3},{p_1})$ de modo que se cumplen las condiciones (2) del $CP$ original:

$$x_{{p_1}{j_0}} + x_{{p_2}{j_0}} \le w_{j_0}$$

$$x_{{p_2}{j_0}} + x_{{p_3}{j_0}} \le w_{j_0}$$

$$x_{{p_3}{j_0}} + x_{{p_1}{j_0}} \le w_{j_0}$$

\medskip
Entonces vale decir lo siguiente:

$$x_{{p_1}{j_0}} + x_{{p_2}{j_0}} \le w_{j_0} < x_{{p_1}{j_0}} + x_{{p_2}{j_0}} + x_{{p_3}{j_0}} + ... + x_{{p_n}{j_0}}$$

Por lo tanto se deduce lo siguiente:

$$x_{{p_3}{j_0}} + ... + x_{{p_n}{j_0}} > 0$$

Podemos suponer, sin p�rdida de generalidad, que $x_{{p_3}{j_0}} = 1$.

\medskip
Por otro lado tenemos que:

$$x_{{p_2}{j_0}} + x_{{p_3}{j_0}} \le w_{j_0}$$

Por condici�n (2) del $LP$ original. Entonces vale lo siguiente:

$$x_{{p_2}{j_0}} + x_{{p_3}{j_0}} \le w_{j_0} < x_{{p_1}{j_0}} + x_{{p_2}{j_0}} + x_{{p_3}{j_0}} + ... + x_{{p_n}{j_0}}$$

Por lo tanto:

$$x_{{p_1}{j_0}} + x_{{p_4}{j_0}} + ... + x_{{p_n}{j_0}} > 0$$

Nuevamente, sin p�rdida de generalidad suponemos que $x_{{p_1}{j_0}} = 1$. Pero se sab�a por lo anterior que:

$$x_{{p_3}{j_0}} + x_{{p_1}{j_0}} \le w_{j_0}$$

Por lo tanto, se llega a que:

$$w_{j_0} \geq 2$$

Lo cual resulta absurdo, pues por (4) $w_j \in \{0, 1\} \forall 1 \le j \le n$. $\qed$

\end{itemize}

Dado que no hay m�s alternativas, la suposici�n de que la desigualdad no se cumple es falsa. $\qed$

\subsection{Desigualdad 2: Agujero de Longitud Impar}

Sea $C_{2k+1}$ el conjunto de v�rtices dentro de un agujero de longitud $2k+1$, con $k >= 2$. Supongamos que la soluci�n no cumple con la desigualdad agujero impar. Por lo tanto podemos decir que $\exists k >= 2$ tal que:

$$
\sum_{p \in C_{2k+1}} x_{p{j_0}} > kw_{j_0}
$$

Con $j_0$ cualquier color tal que $1 \le j_0 \le n$.

\begin{itemize}
\item Supongamos que $w_{j_0} = 0$. Para este caso la desigualdad quedar�a de la siguiente forma:

$$
\sum_{p \in C_{2k+1}} x_{p{j_0}} > 0
$$
      
Como se puede observar, la restricci�n se cumple si existe alg�n nodo $p$ del agujero $C_{2k+1}$ tal que $x_{p{j_0}} = 1$. Dado que no hay nodos aislados, podemos tomar un eje $(p,p')$ de modo que se cumple la condici�n (2) del $CP$ original:

$$x_{{p}{j_0}} + x_{{p'}{j_0}} \le w_{j_0}$$

O sea que

$$x_{{p'}{j_0}} + 1 \le w_{j_0}$$

Pero por (3) $x_{{p'}{j_0}} \in \{0,1\}$, y $w_{j_0} = 0$, o sea que $x_{{p'}{j_0}} + 1 \le 0$ lo cual es absurdo. $\qed$

\item Supongamos ahora que $w_{j_0} = 1$. Para este caso la desigualdad quedar�a de la siguiente forma:

$$
\sum_{p \in C_{2k+1}} x_{p{j_0}} > k
$$
      
La restricci�n entonces se cumple si hay almenos $k+1$ nodos del agujero pintados con el color $j_0$.

Para un circuito de longitud $2k+1$, la m�xima cantidad de nodos pintados del mismo color para que su coloreo sea v�lido es $k$. Como los nodos del agujero conforman un circuito de longitud impar, si pintamos $k+1$ o m�s nodos del mismo color, este coloreo ser�a inv�lido para el mismo circuito, por lo tanto hay almenos dos nodos que son adyacentes entre s� y est�n pintados del mismo color. Por consiguiente el coloreo es inv�lido y la suposici�n de que $w_{j_0} = 1$ es absurda. $\qed$

\end{itemize}

Dado que no hay m�s alternativas, la suposici�n de que la desigualdad no se cumple es falsa. $\qed$

\clearpage

\newpage
\section{Ejercicio 3}

%TODO:hacerlo

\clearpage

\newpage
%TODO: poner las referencias del algoritmo goloso (NEW BEST IN) pra buscar cliques y demas
\begin{thebibliography}{1}

\bibitem{cliquecutalgorithm} \href{http://dx.doi.org/10.1016/j.dam.2005.05.022}{Isabel M�ndez D�az, Paula Zabala. A branch-and-cut algorithm for graph coloring}\\
(http://dx.doi.org/10.1016/j.dam.2005.05.022)
\bibitem{oddcyclecutalgorithm} \href{http://portal.acm.org/citation.cfm?id=159596.159601}{Karla L. Hoffman, Manfred Padberg. Solving Airline Crew Scheduling Problems by Branch-and-Cut}\\
(http://portal.acm.org/citation.cfm?id=159596.159601)
\bibitem{newbestin} \href{http://www.stasbusygin.org/writings/antism.pdf}{Stanislav Busygin. A Simple Clique Camouflaging Against Greedy Maximum Clique Heuristics}\\
(http://www.stasbusygin.org/writings/antism.pdf)

\end{thebibliography}
\clearpage

\label{LastPage}
\end{document}

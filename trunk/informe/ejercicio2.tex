\section{Ejercicio 2}

Para demostrar que las desigualdades son v�lidas, se utilizar� el m�todo del absurdo. La desigualdad es v�lida para el $CP$ original (sin relajaci�n) si la desigualdad se cumple para todos los puntos dentro de la c�psula convexa de soluciones factibles enteras. Por lo tanto el planteo inicial ser� suponer que el punto no cumple con la nueva restricci�n pero s� con las condiciones originales.

Supongamos que existe un coloreo v�lido para el $CP$ original. Por lo tanto existe $W$ = $(w_0, w_1, ... , w_n)$ y $X = \{x_{vj} | v = 0, ..., n \wedge j = 0, ..., n\}$, donde $v$ son los nodos y $j$ son los colores con los que se pintaron los mismos. Entonces para $X$ vale:

\begin{enumerate}
\item $\displaystyle\sum_{j = 1}^{n} x_{pj} = 1$, $\forall$ $p \in V$

\item $x_{pj} + x_{qj} \le w_j$, $\forall$ $(p,q) \in E$, $j = 1,...,n$

\item $x_{pj} \in \{0,1\}$, $\forall$ $p \in V$, $j = 1,...,n$

\item $w_j \in \{0,1\}$, $\forall$ $j = 1,...,n$

\end{enumerate}

\subsection{Desigualdad 1: Clique}

Sea K una Clique del grafo y sea $j_0$ cualquier color tal que $1 \le j_0 \le n$. Supongamos que la soluci�n no cumple con la desigualdad clique, entonces sabemos que vale lo siguiente:

$$\displaystyle\sum_{p \in K} x_{pj_0} > w_{j_0}$$

\begin{itemize}

\item Supongamos que $|K| = 2$. Sean dos nodos cualesquiera $p_1$ y $p_2$ de la Clique, entonces vale que:

$$x_{{p_1}{j_0}} + x_{{p_2}{j_0}} > w_{j_0}$$

Pero por otro lado, como K es una clique, sabemos que existe un eje $({p_1},{p_2})$ de modo que se cumple la condici�n (2) del $CP$ original:

$$x_{{p_1}{j_0}} + x_{{p_2}{j_0}} \le w_{j_0}$$

Pero esto resulta ser absurdo ya que no puede pasar que $A > w_{j_0}$ y que $A \le w_{j_0}$ siendo $A = x_{{p_1}{j_0}} + x_{{p_2}{j_0}}$.$\qed$


\item Supongamos ahora que $|K| > 2$. Sean tres nodos cualesquiera $p_1$, $p_2$ y $p_3$ de la Clique, entonces vale que:

$$x_{{p_1}{j_0}} + x_{{p_2}{j_0}} + x_{{p_3}{j_0}} + ... + x_{{p_n}{j_0}} > w_{j_0}$$

Por otro lado, como K es una clique, sabemos que existen ejes $({p_1},{p_2})$, $({p_2},{p_3})$, $({p_3},{p_1})$ de modo que se cumplen las condiciones (2) del $CP$ original:

$$x_{{p_1}{j_0}} + x_{{p_2}{j_0}} \le w_{j_0}$$

$$x_{{p_2}{j_0}} + x_{{p_3}{j_0}} \le w_{j_0}$$

$$x_{{p_3}{j_0}} + x_{{p_1}{j_0}} \le w_{j_0}$$

\medskip
Entonces vale decir lo siguiente:

$$x_{{p_1}{j_0}} + x_{{p_2}{j_0}} \le w_{j_0} < x_{{p_1}{j_0}} + x_{{p_2}{j_0}} + x_{{p_3}{j_0}} + ... + x_{{p_n}{j_0}}$$

Por lo tanto se deduce lo siguiente:

$$x_{{p_3}{j_0}} + ... + x_{{p_n}{j_0}} > 0$$

Podemos suponer, sin p�rdida de generalidad, que $x_{{p_3}{j_0}} = 1$.

\medskip
Por otro lado tenemos que:

$$x_{{p_2}{j_0}} + x_{{p_3}{j_0}} \le w_{j_0}$$

Por condici�n (2) del $LP$ original. Entonces vale lo siguiente:

$$x_{{p_2}{j_0}} + x_{{p_3}{j_0}} \le w_{j_0} < x_{{p_1}{j_0}} + x_{{p_2}{j_0}} + x_{{p_3}{j_0}} + ... + x_{{p_n}{j_0}}$$

Por lo tanto:

$$x_{{p_1}{j_0}} + x_{{p_4}{j_0}} + ... + x_{{p_n}{j_0}} > 0$$

Nuevamente, sin p�rdida de generalidad suponemos que $x_{{p_1}{j_0}} = 1$. Pero se sab�a por lo anterior que:

$$x_{{p_3}{j_0}} + x_{{p_1}{j_0}} \le w_{j_0}$$

Por lo tanto, se llega a que:

$$w_{j_0} \geq 2$$

Lo cual resulta absurdo, pues por (4) $w_j \in \{0, 1\} \forall 1 \le j \le n$. $\qed$

\end{itemize}

Dado que no hay m�s alternativas, la suposici�n de que la desigualdad no se cumple es falsa. $\qed$

\subsection{Desigualdad 2: Agujero de Longitud Impar}

Sea $C_{2k+1}$ el conjunto de v�rtices dentro de un agujero de longitud $2k+1$, con $k >= 2$. Supongamos que la soluci�n no cumple con la desigualdad agujero impar. Por lo tanto podemos decir que $\exists k >= 2$ tal que:

$$
\sum_{p \in C_{2k+1}} x_{p{j_0}} > kw_{j_0}
$$

Con $j_0$ cualquier color tal que $1 \le j_0 \le n$.

\begin{itemize}
\item Supongamos que $w_{j_0} = 0$. Para este caso la desigualdad quedar�a de la siguiente forma:

$$
\sum_{p \in C_{2k+1}} x_{p{j_0}} > 0
$$
      
Como se puede observar, la restricci�n se cumple si existe alg�n nodo $p$ del agujero $C_{2k+1}$ tal que $x_{p{j_0}} = 1$. Dado que no hay nodos aislados, podemos tomar un eje $(p,p')$ de modo que se cumple la condici�n (2) del $CP$ original:

$$x_{{p}{j_0}} + x_{{p'}{j_0}} \le w_{j_0}$$

O sea que

$$x_{{p'}{j_0}} + 1 \le w_{j_0}$$

Pero por (3) $x_{{p'}{j_0}} \in \{0,1\}$, y $w_{j_0} = 0$, o sea que $x_{{p'}{j_0}} + 1 \le 0$ lo cual es absurdo. $\qed$

\item Supongamos ahora que $w_{j_0} = 1$. Para este caso la desigualdad quedar�a de la siguiente forma:

$$
\sum_{p \in C_{2k+1}} x_{p{j_0}} > k
$$
      
La restricci�n entonces se cumple si hay almenos $k+1$ nodos del agujero pintados con el color $j_0$.

Para un circuito de longitud $2k+1$, la m�xima cantidad de nodos pintados del mismo color para que su coloreo sea v�lido es $k$. Como los nodos del agujero conforman un circuito de longitud impar, si pintamos $k+1$ o m�s nodos del mismo color, este coloreo ser�a inv�lido para el mismo circuito, por lo tanto hay almenos dos nodos que son adyacentes entre s� y est�n pintados del mismo color. Por consiguiente el coloreo es inv�lido y la suposici�n de que $w_{j_0} = 1$ es absurda. $\qed$

\end{itemize}

Dado que no hay m�s alternativas, la suposici�n de que la desigualdad no se cumple es falsa. $\qed$

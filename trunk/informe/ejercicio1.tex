\section{Ejercicio1}

El objetivo de este ejercicio consiste en poner a prueba el problema de coloreo, variando algunos de los par�metros que $Cplex$ utiliza para resolver el problema. Los mismos son par�metros gen�ricos que forman parte del comportamiento que tiene $Cplex$ para computar la soluci�n.

En este caso en particular, se tomaron par�metros que se utilizan en el m�todo de Branch and Bound. La prueba entonces, consiste en modificar los valores de dichos par�metros y observar como es el desempe�o del proceso, en tiempo requerido y en cantidad de nodos creados por $Cplex$, hasta obtener la soluci�n del problema.

Los par�metros elegidos son los siguientes:

\begin{itemize}
\item \textbf{CPX\_PARAM\_BRDIR:} Este par�metro sirve para modificar la direcci�n en la cual se har� en branching en cada paso. Los posibles valores para el mismo son los siguientes:
   \begin{itemize}
   \item CPX\_BRDIR\_UP, que indica que el branching siempre debe hacerse por la parte superior.
   \item CPX\_BRDIR\_DOWN, que indica que el branching debe hacerse por la parte inferior.
   \item CPX\_BRDIR\_AUTO, que indica que $Cplex$ decidir� que camino tomar.
   \end{itemize}

\item \textbf{CPX\_PARAM\_LBHEUR:} Indica si las heur�sticas locales en cada branching est�n o no activadas. Los valores para el mismo son: CPX\_ON y CPX\_OFF.

\item \textbf{CPX\_PARAM\_VARSEL:} CPX\_PARAM\_LBHEUR depende de este par�metro, que indica que estrategia usar para seleccionar una variable antes de realizar el branching. Los posibles valores son:
    \begin{itemize}
    \item CPX\_VARSEL\_MININFEAS, que realiza el branch con la variable con inviabilidad m�nima (es decir, la variable fraccionaria mas cercana a alg�n valor entero).
    \item CPX\_VARSEL\_MAXINFEAS, que realiza el braching con la variable mas inviable(es decir la mas lejana a un valor entero)
    \item CPX\_VARSEL\_PSEUDO, la variable es elegida a trav�s de pseudo costos.
    \item CPX\_VARSEL\_PSEUDO, elije la variable a trav�s de la resoluci�n de diferentes sub problemas que permiten saber cuan prometedora es la elecci�n de dicha variable.
    \item CPX\_VARSEL\_PSEUDOREDUCED, selecciona la variable basado en costos.
    \end{itemize}

\item \textbf{CPX\_PARAM\_STRONGITLIM:} Indica el l�mite de iteraciones a realizar en cada una de las variables candidatas para realizar el branching. Los valores que puede tomar son: 0, automatico o un n�mero positivo, que indica las iteraciones fijas a realizar.

\item \textbf{CPX\_PARAM\_ZEROHALFCUTS:} Aqu� se indica si se har�n o no cortes de tipo parte entera hacia abajo para los valores de algunas de las restricciones. Los posibles valores son: -1, desactivado, 0 automatico, 1 normal y 2 agresivo.
\end{itemize}

\subsection{Pruebas realizadas}


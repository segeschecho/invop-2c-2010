\section{Conclusiones}

Este trabajo pr�ctico fue �til para aprender sobre $CPLEX$ y experimentar con problemas de coloreo de grafos modelados como problemas de Programaci�n Lineal Entera. Se Pudo apreciar la eficiencia de $CPLEX$ con respecto a sus resultados, sus tiempos y la muy variada forma de configurarlo para lograr su m�xima performance para cada problema. Se uudo experimentar y aportarle a $CPLEX$ nuevas heur�sticas de separaci�n y heur�sticas iniciales de ideas simples pero no por eso poco eficientes, por lo que estas tomaron un rol muy importante a la hora de reducir el tiempo de ejecuci�n del algoritmo.

Fue muy interesante investigar, pensar e implementar las heur�sticas de separaci�n y las iniciales, y poder apreciar su comportamiento comparando distintas configuraciones del sistema. Sin embargo las infinitas posibilidades de configurarlo sobrepasaron nuestro tiempo disponible y nos deja con inter�s de seguir probando otras configuraciones con m�s clases de tests.

Se not� que una heur�stica o combinaci�n de par�metros que pueden ser muy buenos en la practica con ciertas instancias de prueba, puede resultar muy mala con otras. Esto da a pensar que todav�a se puede mejorar a�n mas la performance del programa, analizando mejor como pueden llegar a ser las instancias de prueba.

Se pudo apreciar mejor la importancia de tener buenas heur�sticas iniciales, que permitan reducir el problema o acotar el rango de posibles soluciones, para liberar de trabajo $Cplex$ y de esta forma reducir los tiempos ejecuci�n, ya que se reduce el espacio de b�squeda. Esto se pudo observar en las pruebas realizadas, sobre todo en el ejercicio n�mero uno, donde los tiempos usando heur�sticas iniciales result� ser abruptamente menor a los tiempos corriendo las mismas instancias pero sin ning�n tipo de heur�stica inicial.

Adem�s de las heur�sticas iniciales, los cortes espec�ficos que se implementaron, mostraron que si se analiza mejor el tipo de problema, se pueden crear metodolog�as, o cortes en este caso, que mejoren la performance considerablemente y aprovechen mejor los tipos de instancia que pueden llegar a tocar.
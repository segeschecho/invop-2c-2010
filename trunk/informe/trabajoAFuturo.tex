\section{Trabajo a futuro}

Una de las cosas que queda pendiente mejorar, es la complejidad de las heur�sticas iniciales, ya que pueden ser mejoradas en ese aspecto. Adem�s quedaron algunas ideas para mejorar la eficiencia de las heur�sticas, en las que se destacan, la exploraci�n de mas de un nodo inicial o recorrer el grafo de forma diferente.

Con respecto a las heur�sticas de separaci�n, es cierto que se puede mejorar tambi�n la complejidad de cada una ya que si bien logran un buen promedio de cortes, muchas veces no mejoran demasiado el tiempo de ejecuci�n, y adem�s se puede mejorar la eficiencia o realizar otros tipos de heur�sticas, aunque esto �ltimo requiera mas tiempo para analizar el problema.

Queda pendiente continuar con las pruebas de configuraciones de los algoritmos para buscar la m�xima performance del problema.

El modelo para realizar este trabajo no fue modificado con respecto al modelo original del trabajo pr�ctico. Quedar�a entonces pendiente ver qu� resultados se pueden obtener cambiando el mismo, ya sea agregando restricciones para forzar el orden de los colores o para describir alguna otra cualidad del problema, cualidades que no se analizaron a fondo por el tiempo y los requerimientos del trabajo. Agregando nuevas restricciones al problema podr�a hallarse nuevas formas de realizar cortes, que puedan mejorar a�n m�s la performance.

Queda bastante claro entonces que se pueden hacer y mejorar muchas cosas y que el problema de coloreo, visto como un problema de optimizaci�n combinatoria, est� bastante abierto en cuanto a nuevas ideas para resolver de manera m�s eficiente el problema.
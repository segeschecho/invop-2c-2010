\section{Ejercicio 4}

En esta secci�n se har�n comparaciones entre los algoritmos de las secciones anteriores con un algoritmo que corre $Cplex$ standard con los par�metros por defecto. Adem�s, se analizar� que aportes se agregan al sumar los cortes programados especialmente a los cortes que trae $Cplex$ por defecto.

Los an�lisis y las comparaciones se har�n en base a una serie de tipos de tests. que tienen diferentes configuraciones entre los cortes a usar, las heur�sticas iniciales y diferentes par�metros que definen el comportamiento del branching. Los tipos de test, son los siguientes:

\begin{enumerate}
\item Tipo 1: Se setean las heur�sticas iniciales propias, sin activar las heur�sticas iniciales de $Cplex$. Se desactivan todos los tipos de cortes. Se setean los par�metros de Branching seg�n el tipo 7 de los tipos del ejercicio n�mero 1.

\item Tipo 2: Se setean las heur�sticas iniciales propias, sin activar las de $Cplex$. Se activan los cortes de $Cplex$ (Solo los de $Cplex$), Se setean los par�metros de Branching seg�n el tipo 7 de los tipos del ejercicio n�mero 1.

\item Tipo 3: Se setean las heur�sticas iniciales propias, sin activar las de $Cplex$. Se activan los cortes de $Cplex$ y los cortes Clique y Agujero Impar implementados, Se setean los par�metros de Branching seg�n el tipo 7 de los tipos del ejercicio n�mero 1.

\item Tipo 4: Se setean las heur�sticas iniciales propias, sin activar las de $Cplex$. Se activan los cortes de $Cplex$ y los cortes Clique y Agujero Impar implementados, Se setean los par�metros de Branching de forma autom�tica, dejando que $Cplex$ elija el apropiado seg�n el caso.

\item Tipo 5: Solo $Cplex$. Se setean las heur�sticas iniciales de $Cplex$ . Se activan los cortes de $Cplex$. Se setean los par�metros de Branching de forma autom�tica, dejando que $Cplex$ elija el apropiado seg�n el caso.

\end{enumerate}

Con estos tipos de tests, se espera ver como se comportan las heur�sticas implementadas versus las heur�sticas iniciales de $Cplex$, el rendimiento en tiempo y cantidad de nodos entre $Cplex$ autom�tico versus las variantes que se ten�an en secciones anteriores.